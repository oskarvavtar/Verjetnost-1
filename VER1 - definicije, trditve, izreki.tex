\documentclass[11pt]{article}
\usepackage[utf8]{inputenc}
\usepackage[slovene]{babel}

\usepackage{amsthm}
\usepackage{amsmath, amssymb, amsfonts}

\theoremstyle{definition}
\newtheorem{definicija}{Definicija}[section]

\newtheorem{lema}{Lema}
\newtheorem{trditev}{Trditev}
\newtheorem{izrek}{Izrek}
\newtheorem*{dokaz}{Dokaz}

\title{Verjetnost 1 - definicije, trditve in izreki}
\author{Oskar Vavtar}
\date{2020/21}

\begin{document}
\maketitle
\pagebreak
\tableofcontents
\pagebreak

\section{Neformalni uvod v verjetnost}

\begin{definicija}[Verjetnost]

Izvajamo poskus. Opazujemo določen pojav, ki ga imenujemo \textit{dogodek}. Poskus ponovimo $n$ - krat. \\
Definirajmo \textit{frekvenco dogodka} $k_n(A)$ kot število ponovitev, pri katerih se dogodek zgodi. \\
\textit{Relativna frekvenca} je definirana kot $f_n(A) = \frac{k_n(A)}{n}$.
Zaporedje $\{f_n(A)\}_{n \in \mathbb{N}}$ konvergira k nekem številu $p \in [0, 1].$ \\
\textsc{Statistična definicija verjetnosti} je definirana kot
$$\mathbb{P}(A) = \lim_{n\to\infty} \{f_n(A)\}.$$
\textsc{Klasična definicija verjetnosti} je definirana kot
$$\mathbb{P}(A) = \frac{\text{št. ugodnih izidov A}}{\text{št. vseh izidov A}}.$$
Če je izidov neskončno mnogo uporabimo geometrijsko definicijo verjetnosti.

\end{definicija}

\end{document}