\documentclass[11pt]{article}
\usepackage[utf8]{inputenc}
\usepackage[slovene]{babel}
\usepackage{relsize}
\usepackage{amsthm}
\usepackage{amsmath, amssymb, amsfonts}

\theoremstyle{definition}
\newtheorem{definicija}{Definicija}[section]

\theoremstyle{definition}
\newtheorem{trditev}{Trditev}[section]

\theoremstyle{definition}
\newtheorem{izrek}{Izrek}[section]

\newtheorem*{posledica}{Posledica}
\newtheorem*{opomba}{Opomba}
\newtheorem{lema}{Lema}
\newtheorem*{dokaz}{Dokaz}
\newtheorem*{posplošitev}{Posplošitev}

\title{Verjetnost 1 - definicije, trditve in izreki}
\author{Oskar Vavtar}
\date{2020/21}

\begin{document}
\maketitle
\pagebreak
\tableofcontents
\pagebreak

% #################################################################################################

\section{DEFINICIJA VERJETNOSTI}

% *************************************************************************************************

\subsection{Neformalni uvod v verjetnost}

\begin{definicija}[Verjetnost]

Izvajamo poskus. Opazujemo določen pojav, ki ga imenujemo \textit{dogodek}. Poskus ponovimo $n$ - krat. \\
Definirajmo \textit{frekvenco dogodka} $k_n(A)$ kot število ponovitev, pri katerih se dogodek zgodi. \\
\textit{Relativna frekvenca} je definirana kot $f_n(A) = \frac{k_n(A)}{n}$.
Zaporedje $\{f_n(A)\}_{n \in \mathbb{N}}$ konvergira k nekem številu $p \in [0, 1].$ \\
\textsc{Statistična definicija verjetnosti} je definirana kot
$$\mathbb{P}(A) ~=~ \lim_{n\to\infty} \{f_n(A)\}.$$
\textsc{Klasična definicija verjetnosti} je definirana kot
$$\mathbb{P}(A) ~=~ \frac{\text{št. ugodnih izidov A}}{\text{št. vseh izidov A}}.$$
Če je izidov neskončno mnogo uporabimo geometrijsko definicijo verjetnosti.

\end{definicija}
\vspace{0.5cm}

% *************************************************************************************************

\subsection{Aksiomična definicija verjetnosti}
\vspace{0.5cm}

\begin{definicija}

Imamo \textit{prostor vseh izidov} oz. \textit{vzorčni prostor} $\Omega$. \textit{Dogodki} so nekatere (ne nujno vse) podmnožice $\Omega$.

\end{definicija}
\vspace{0.5cm}

\begin{definicija}[Operacije na dogodkih]
~
\begin{enumerate}
	\item \textsc{Vsota} oz. \textsc{unija} dogodkov:
	$$A + B ~=~ A \cup B$$
	je dogodek, ki se zgodi, če se zgodi vsaj eden od dogodkov $A$ in $B$. \\
	\item \textsc{Produkt} oz. \textsc{presek} dogodkov:
	$$A \cdot B ~=~ A \cap B$$
	je dogodek, ki se zgodi, če se zgodita oba dogodka $A$ in $B$ hkrati.
	\item \textsc{Nasprotni} dogodek oz. \textsc{komplement} dogodkov:
	$$\bar{A} ~=~ A^C$$
	je dogodek, ki se zgodi, če se dogodek $A$ ne zgodi.
\end{enumerate}
\end{definicija}
\vspace{0.5cm}

\begin{opomba}

Pravila za računanje z dogodki:
\begin{enumerate}
	\item Idempotentnost: 
	$$A \cup A ~=~ A ~=~ A \cap A$$
	\item Komutativnost:
	$$A \cup B ~=~ B \cup A$$
	$$A \cap B ~=~ B \cap A$$
	\item Asociativnost:
	$$(A \cup B) \cup C ~=~ A \cup (B \cup C)$$
	$$(A \cap B) \cap C ~=~ A \cap (B \cap C)$$
	\item Distributivnost:
	$$(A \cup B) \cap C ~=~ (A \cap C) \cup (B \cap C)$$
	$$(A \cap B) \cup C ~=~ (A \cup C) \cap (B \cup C)$$
	\item de Morganova zakona:
	$$(A \cap B)^C ~=~ A^C \cup B^C$$
	$$(A \cup B)^C ~=~ A^C \cap B^C$$
	Še več:
	$$\left( \bigcup_i A_i \right)^C ~=~ \bigcap_i {A_i}^C$$
	$$\left( \bigcap_i A_i \right)^C ~=~ \bigcup_i {A_i}^C$$
\end{enumerate}
\end{opomba}
\vspace{0.5cm}

\begin{opomba}

V splošnem ni vsaka podmnožica $A \subset \Omega$ dogodek.

\end{opomba}
\vspace{0.5cm}

\begin{definicija}[$\sigma$-algebra]

\textit{Neprazna} družina podmnožic (dogodkov) $\mathcal{F}$ v $\Omega$ je $\sigma$-algebra, če velja:
\begin{enumerate}
	\item \textit{Zaprtost komponente}:
	$$A \in \mathcal{F} ~\Rightarrow~ A^C \in \mathcal{F}$$
	\item \textit{Zaprtost števne unije}:
	$$A_1, A_2, A_3, \ldots \in \mathcal{F} ~\Rightarrow~ \bigcup_{i=1}^{\infty} A_i \in \mathcal{F}$$
\end{enumerate}
\end{definicija}
\vspace{0.5cm}

\begin{opomba}

Če v (2) zahtevamo manj:
$$A, B \in \mathcal{F} ~\Rightarrow~ A \cup B \in \mathcal{F},$$
je $\mathcal{F}$ \textit{algebra}. V algebri imamo torej zaprtost za končne unije in končne preseke, medtem ko je $\sigma$-algebra zaprta celo za števne preseke.

\end{opomba}
\vspace{0.5cm}

\begin{definicija}
Naj bo $\mathcal{F}$ $\sigma$-algebra, $\Omega$ vzorčni prostor. \textit{Verjetnostna mera} na $(\mathcal{F}, \Omega$), je preslikava $\mathbb{P}: \mathcal{F} \rightarrow \mathbb{R}$ za katero velja:
\begin{enumerate}
	\item $\mathbb{P} \geq 0$, za $\forall A \in \mathbb{F}$
	\item $\mathbb{P}(\Omega) = 1$
	\item Za poljubne paroma nerazdružljive dogodke velja
	$$\mathbb{P}\left( \bigcup_{i=1}^{\infty} A_i \right) ~=~ \sum_{n=1}^{\infty} \mathbb{P}(A_i)$$
\end{enumerate}
Trojico $(\mathcal{F}, \Omega, \mathbb{P})$ imenujemo \textit{verjetnostni prostor}.
\end{definicija}
\vspace{0.5cm}

\begin{posledica}[Posledice verjetnostnih aksiomov]
~
\begin{enumerate}

\item[(a)] $\mathbb{P}(\emptyset) = 0$ \\
\textbf{Dokaz:} v (\textit{3.}) vzamemo $A_i = \emptyset$: 
$\mathbb{P}(\emptyset) = \mathbb{P}(\emptyset) + \mathbb{P}(\emptyset) + \mathbb{P}(\emptyset) + \ldots$

\item[(b)] $\mathbb{P}$ je končno aditivna, torej za končno mnogo paroma nerazdružljivih dogodkov velja: \\
$\mathbb{P}(A_1 \cup \ldots \cup A_n) = \mathbb{P}(A_1) + \ldots + \mathbb{P}(A_n)$ \\
\textbf{Dokaz:} v (\textit{3.}) vzamemo $A_{n+1} = A_{n+2} = \ldots = \emptyset$ in uporabimo (\textit{a})

\item[(c)] $\mathbb{P}$ je \textit{monotona}, torej $A \subseteq B \Rightarrow \mathbb{P}(A) \subseteq \mathbb{P}(B)$ \\
Še več: $A \subseteq B \Rightarrow \mathbb{P}(B \setminus A) = \mathbb{P}(B) - \mathbb{P}(A)$ \\
\textbf{Dokaz:} ker je $B = A \cup (B-A)$, $A \cap (B \setminus A) = \emptyset$, zaradi (\textit{b}) velja $\mathbb{P}(B) = \mathbb{P}(A) + \mathbb{P}(B-A)$

\item[(d)] $\mathbb{P}(A^C) = 1 - \mathbb{P}(A)$ \\
\textbf{Dokaz:} v (\textit{c}) vzamemo $B = \Omega$

\item[(e)] $\mathbb{P}$ je zvezna:
\begin{enumerate}
	\item[(i)] $A_1 \subseteq A_2 \subseteq A_3 \subseteq \ldots \Rightarrow \mathbb{P}(\mathlarger{\bigcup_{i=1}^{\infty} A_i}) = \lim_{n \rightarrow \infty} \mathbb{P}(A_n)$
	\item[(ii)] $B_1 \supseteq B_2 \supseteq B_3 \supseteq \ldots \Rightarrow \mathbb{P}(\mathlarger{\bigcap_{i=1}^{\infty} B_i}) = \lim_{n \rightarrow \infty} \mathbb{P}(B_n)$
\end{enumerate} 
\textbf{Dokaz:}
\begin{enumerate}
	\item[(i)] Definiramo: $C_i = A_i \setminus A_{i-1}$ za $i = 2, 3, \ldots$, $C_1 = A_1$ \\
	Potem je $A_n = C_1 \cup \ldots \cup C_n$, $~C_i \cap C_j = \emptyset$ za $i \neq j$, \\ 
	$\mathlarger{\bigcup_{i=1}^{\infty} A_i} = \mathlarger{\bigcup_{i=1}^{\infty} C_i}$ \\
	Torej je 
	\begin{align*}
	\mathbb{P}(\bigcup_{i=1}^{\infty} A_i ~=~ \mathbb{P}(\bigcup_{i=1}^{\infty} C_i) ~&=~ \sum_{i=1}^{\infty} \mathbb{P}(C_i) ~= \\
	=~ \lim_{n \rightarrow \infty} \sum_{i=1}^{n} \mathbb{P}(C_i) ~&=~ \lim_{n \rightarrow \infty} \mathbb{P}(\bigcup_{i=1}^{n} C_i) ~=~ \lim_{n \rightarrow \infty} \mathbb{P}(A_n) \\
	\end{align*}
	\item[(ii)] Ker je $B_1 \supseteq B_2 \supseteq B_3 \supseteq \ldots$, sledi ${B_1}^C \subseteq {B_2}^C \subseteq {B_3}^C \subseteq \ldots$ \\
	Po (\textit{i}) je $\mathbb{P}(\bigcup_{i=1}^{\infty} {B_i}^C) = \mathlarger{\lim_{i \rightarrow \infty} \mathbb{P}({D_i}^C)}$, toda $\mathlarger{\bigcup_{i=1}^{\infty} {B_i}^C = \left( \bigcap_{i=1}^{\infty} B_i \right)^C}$. \\
	Zato je $1 - \mathbb{P}(\bigcap_{i=1}^{\infty} B_i) = \mathlarger{\lim_{i \rightarrow \infty}} (1 - \mathbb{P}(B_i))$, od koder sledi želena neenakost.
\end{enumerate}
\end{enumerate}
\end{posledica}
\vspace{0.5cm}

% *************************************************************************************************

\pagebreak

% #################################################################################################

\section{POGOJNA VERJETNOST}
\vspace{0.5cm}

\begin{definicija}[Pogojna verjetnost]

\textit{Pogojna verjetnost} dogodka $A$ glede na dogodek $B$, $\mathbb{P}(A | B)$, je verjetnost dogodka $A$ če vemo, da se je zgodil dogodek $B$.
Posplošimo:
$$\mathbb{P}(A | B) ~=~ \frac{\mathbb{P}(A \cap B)}{\mathbb{P}(B)}$$

\end{definicija}
\vspace{0.5cm}

\begin{posledica}

Iz definicije sledi
$$\mathbb{P}(A \cap B) ~=~ \mathbb{P}(A | B) \cdot \mathbb{P}(B).$$
Če posplošimo na $n$ dogodkov dobimo
$$\mathbb{P}(A_1 \cap A_2 \cap \ldots \cap A_n) = \mathbb{P}(A_1) \cdot \mathbb{P}(A_2 | A_1) \cdot \ldots \cdot \mathbb{P}(A_n | A_1 \cap \ldots \cap A_{n-1}).$$
Če velja $\mathbb{P}(A) = \mathbb{P}(A | B)$, sta dogodka neodvisna.

\end{posledica}
\vspace{0.5cm}

\begin{izrek}[Izrek o popolni verjetnosti]

Naj bo $(H_i)_i$ popoln sistem dogodkov. Potem je 
$$A = A \cap \Omega ~=~ A \cap {\left( \bigcup_i H_i \right)} ~=~ \bigcup_i A \cap H_i$$
in iz tega sledi
$$\mathbb{P}(A) ~=~ \sum_i \mathbb{P}(A \cap H_i) ~=~ \sum_i \mathbb{P}(H_i) \cdot \mathbb{P}(A | H_i).$$
To je formula za \textit{popolno verjetnost}.

\end{izrek}
\vspace{0.5cm}

\begin{posledica}[Bayesova formula]

Iz definicije pogojne verjetnosti vemo
$$\mathbb{P}(H_i | A) ~=~ \frac{\mathbb{P}(H_i \cap A}{\mathbb{P}(A)} ~=~ \frac{\mathbb{P}(H_i) \cdot \mathbb{P}(A | H_i)}{\mathbb{P}(A)}.$$
Če v imenovalec vstavimo izrek o popolni verjetnosti, dobimo \textit{Bayesovo formulo}:
$$\mathbb{P}(H_i | A) ~=~ \frac{\mathbb{P}(H_i) \cdot \mathbb{P}(A | H_i)}{\mathbb{P}(H_1) \cdot \mathbb{P}(A | H_1) + \ldots + \mathbb{P}(H_n) \cdot \mathbb{P}(A | H_n)} ~=~ \frac{\mathbb{P}(H_i) \cdot \mathbb{P}(A | H_i)}{\sum_i \mathbb{P}(H_i) \cdot \mathbb{P}(A | H_i)}$$

\end{posledica}
\vspace{0.5cm}

\begin{definicija}[Neodvisnost $2$ dogodkov]

Dogodka $A$ in $B$ sta \textit{neodvisna}, če je 
$$\mathbb{P}(A \cap B) ~=~ \mathbb{P}(A) \cdot \mathbb{P}(B).$$
Če je $\mathbb{P}(B) > 0$ to enakost lahko zapišemo kot
$$\mathbb{P}(A) ~=~ \frac{\mathbb{P}(A \cap B)}{\mathbb{P}(B)} ~=~ \mathbb{P}(A | B).$$

\end{definicija}
\vspace{0.5cm}

\begin{definicija}[Neodvisnost $k$ dogodkov]

Dogodki $(A_i)_i$ so \textit{neodvisni}, če za poljuben končen nabor različnih dogodkov $A_{i_1}, A_{i_2}, \ldots, A_{i_k}$ velja
$$\mathbb{P}(A_{i_1} \cap A_{i_2} \cap \ldots \cap A_{i_k}) ~=~ \mathbb{P}(A_{i_1}) \cdot \mathbb{P}(A_{i_2}) \cdot \ldots \cdot \mathbb{P}(A_{i_k}).$$
Če zahtevamo le za $k = 2$, torej $A_i$ in $A_j$ sta neodvisna le za $i \neq j$, potem rečemo, da so dogodki \textit{paroma neodvisni}. To je šibkejši pogoj kot neodvisnost.

\end{definicija}

\begin{trditev}

Če sta dogodka $A$ in $B$ neodvisna, potem sta neodvisna tudi dogodka $A^C$ in $B$, $A$ in $B^C$ ter $A^C$ in $B^C$.

\end{trditev}
\vspace{0.5cm}

% *************************************************************************************************

\pagebreak

% #################################################################################################

\section{ZAPOREDJA NEODVISNIH PONOVITEV \\POSKUSA}
\vspace{0.5cm}

\begin{definicija}[Bernoullijeva formula]

Imejmo zaporedje $n$ neodvisnih ponovitev poskusa, določenega z verjetnostnim prostorom $(\Omega, \mathcal{F}, p)$, v katerem je možen dogodek $A$ s \\$\mathbb{P}(A) = p$.
Z $A_n(k)$ označimo dogodek, da se $A$ zgodi natanko $k$-krat, $k = 0, 1, 2, \ldots, n$. \\

\noindent $A_n(k)$ je \textit{disjunktivna unija} $\mathlarger{\binom{n}{k}}$ dogodkov, da se $A$ zgodi na predpisanih $k$ mestih; na ostalih pa $A^C$. Verjetnost le teh dogodkov je $p^k \cdot q^{n-k}$. Zato velja \textit{Bernoullijeva formula}
$$\mathbb{P}_n(k) ~=~ \binom{n}{k} \cdot p^k \cdot q^{n-k}.$$

\end{definicija}
\vspace{0.5cm}

\begin{trditev}[Aproksimativni formuli za $\mathbb{P}_n(k)$]

~\\
\begin{enumerate}
	\item[a)] \textsc{Poissonova formula}: če je $p$ blizu $0$ in $n$ velik, potem je
	$$\mathbb{P}_n(k) ~\approx~ \frac{\lambda^k}{k!} \cdot e^{-\lambda}, ~~~\text{kjer je}~ \lambda = n \cdot p$$
	\item[b)] \textsc{Laplaceova lokalna formula}: za velike $n$ velja
	$$\mathbb{P}_n(k) ~\approx~ \frac{1}{\sqrt{2 \pi n p q}} \cdot e^{-\frac{(k-np)^2}{2npq}}$$
\end{enumerate}

\end{trditev}
\vspace{0.5cm}

% *************************************************************************************************

\pagebreak

% #################################################################################################

\section{SLUČAJNE SPREMENLJIVKE}
\vspace{0.5cm}

\begin{definicija}

\textit{Realna slučajna spremenljivka} na verjetnostnem prostoru $(\Omega, \mathcal{F}, p)$ je funkcija $X: \Omega \rightarrow \mathbb{R}$ z lastnostjo, da je za $\forall x \in \mathbb{R}$ množica $\{ \omega \in \Omega \mid X(\omega) \leq x \}$ v $\mathcal{F}$, se pravi je dogodek.\\

\noindent Oznaka:
$$\{ \omega \in \Omega \mid X(\omega) \leq x \} ~\equiv~ X^{-1}((-\infty, x]) ~\equiv~ (X \leq x)$$

\end{definicija}
\vspace{0.5cm}

\begin{definicija}[Porazdelitvena funkcija slučajne spremenljivke]

Funkcija $F_X: \mathbb{R} \rightarrow \mathbb{R}$, definirana s $F_X(x) = \mathbb{P}(X \leq x)$ se imenuje \textit{porazdelitvena funkcija slučajne spremenljivke $x$}.

\end{definicija}
\vspace{0.5cm}

\begin{trditev}[Lastnosti porazdelitvene funkcije $F = F_X$]
~\\
\begin{enumerate}
	\item $0 \leq F(x) \leq 1$ za $\forall x \in \mathbb{R}$
	\item  $F$ je naraščajoča funkcija:
	$$x_1 < x_2 ~\Rightarrow~ F(x_1) \leq F(x_2)$$
	\item $\mathlarger{\lim_{x \rightarrow \infty} F(x) = 1}$, $\mathlarger{\lim_{x \rightarrow -\infty} F(x) = 0}$
	\item $F$ je \textit{z desne zvezna}, torej $F(x+) = F(x)$ za $\forall x \in \mathbb{R}$, kjer je $\mathlarger{F(x+) = \lim_{h \searrow 0} F(x+h)}$ desna limita.
	\item $F(x-) = \mathbb{P}(X < x)$ 
	\begin{align*}
	\mathbb{P}(x_1 < X \leq x_2) ~&=~ F(x_2) - F(x_1) \\
	\mathbb{P}(x_1 < X < x_2) ~&=~ F(x_2-) - F(x_1) \\
	\mathbb{P}(x_1 \leq X \leq x_2) ~&=~ F(x_2) - F(x_1-) \\
	\mathbb{P}(x_1 \leq X < x_2) ~&=~ F(x_2-) - F(x_1-) 
	\end{align*}
\end{enumerate}
\end{trditev}
\vspace{0.5cm}

\begin{definicija}[Diskretne slučajne spremenljivke]

Slučajna spremenljivka je \textit{diskretno porazdeljena}, če je njena zaloga vrednosti \textit{končna} ali \textit{števna končna} množica. Naj bo $\{ x_1, x_2, \ldots \}$ zaloga vrednosti. Vpeljemo verjetnostno funkcijo $p_k = \mathbb{P}(X = x_k)$, $k = 1, 2, 3, \ldots.$ Ker so $\{ (X = x_k) \}_k$ \textit{poln} sistem dogodkov, je 
$$\sum_k p_k ~=~ 1.$$
$X$ lahko zapišemo s shemo 
$X: \begin{pmatrix}
	x_1 & x_2 & \cdots & x_n \\
	p_1 & p_2 & \cdots & p_n
\end{pmatrix}$. Porazdelitvena funkcija:
$$F(x) ~=~ \mathbb{P}(X \leq x) ~=~ \mathbb{P}\left( \bigcup_{k ~\mid~ x_k \leq x} (X = x_k) \right) ~=~ \sum_{k ~\mid~ x_k \leq x} p_k.$$ \\

\noindent Pogoste diskretne porazdelitve:
\begin{enumerate}
	\item \textsc{Enakomerna porazdelitev} na $n$ točkah $x_1, \ldots, x_n$:
	$$X: ~\begin{pmatrix}
		x_1 & x_2 & \cdots & x_n \\
		\frac{1}{n} & \frac{1}{n} & \cdots & \frac{1}{n}
	\end{pmatrix}$$
	
	\item \textsc{Bernoullijeva porazdelitev}, $Ber(p)$, ~$p \in (0, 1)$:
	$$X: ~\begin{pmatrix}
		0 & 1 \\
		(1-p) & p
	\end{pmatrix}$$
	Indikatorska funkcija:
	$$1_A(\omega) = \begin{cases}
	1 ; ~\omega \in A \\
	0 ; ~\omega \notin A
	\end{cases}$$
	
	\item \textsc{Binomska porazdelitev}, $Bin(n, p)$, ~$n \in \mathbb{R}$, ~$p \in (0, 1)$:
	$$X: ~\begin{pmatrix}
		0 & 1 & 2 & \cdots & n \\
		p_0 & p_1 & p_2 & \cdots & p_n
	\end{pmatrix},$$
	kjer za $k = 0, 1, \ldots, n$ velja
	$$p_k ~=~ \mathbb{P}(X = k) ~=~ \binom{n}{k} p^k (1 - p)^{n-k}.$$
	
	\item \textsc{Poisonnova porazdelitev}, $Poi(\lambda)$, ~$\lambda = n \cdot p > 0$:
	$$p_k ~=~ \mathbb{P}(X = k) ~=~ \mathbb{P}_n(k) ~=~ \frac{\lambda^k}{k!} \cdot e^{-\lambda}, ~~~k = 0, 1, \ldots$$
	Sledi:
	$$\sum_{k=0}^{\infty} ~=~ e^{-\lambda} \sum_{k=0}^{\infty} \frac{\lambda^k}{k!} ~=~ e^{-\lambda} \cdot e^{\lambda} ~=~ 1.$$
	
	\item \textsc{Geometrijska porazdelitev}: $Geo(p), ~p \in (0, 1)$.\\
	$(X = k)$ je dogodek, da se $A$ zgodi prvič v $k$-ti ponovitvi:
	$$p_k ~=~ \mathbb{P}(X = k) ~=~ p \cdot q^{k-1}, ~~~k = 1, 2, 3, \ldots$$
	Torej $\underbrace{A^C \cdot \ldots \cdot A^C}_{k-1} \cdot A = \underbrace{q \cdot \ldots \cdot q}_{k-1} \cdot p = p \cdot q^{k-1}.$ Velja:
	$$\sum_{k=1}^{\infty} p_k ~=~ p \sum_{k=1}^{\infty} q^{k-1} ~=~ \frac{p}{1-q} = 1$$
	
	\item \textsc{Pascalova porazdelitev}: $Pas(m, p), ~m \in \mathbb{N}, ~p \in (0, 1).$\\
	$(X = k)$ je dogodek, da se $A$ zgodi $m$-tič pri $k$-ti ponovitvi:
	$$p_k ~=~ \binom{k-1}{m-1} \cdot p^m \cdot q^{k-m}, ~~~k = m, m+1, \ldots$$
	
	\item \textsc{Hipergeometrijska porazdelitev}: $Hip(n; M, N), ~n \leq \min{\{M, N-M\}}$\\
	V posodi je $M$ belih in $(N-M)$ črnih kroglic. Slučajno izvlečemo $n$ kroglic. $X$ naj pomeni število belih kroglic med izvlečenimi, kjer za $k = 0, 1, \ldots, n$ velja:
	$$p_k ~=~ \mathbb{P}(X = k) ~=~ \frac{\binom{M}{k} \binom{N-M}{n-k}}{\binom{N}{n}}$$
	Ker je $\left[(X = k)\right]_{k=0}^n$ popoln sistem dogodkov je $\mathlarger{\sum_{k=0}^n p_k = 1}$, torej je
	$$\sum_{k=0}^n \binom{M}{n} \cdot \binom{N-M}{n-k} ~=~ \binom{N}{n}.$$
\end{enumerate}

\end{definicija}
\vspace{0.5cm}

\begin{definicija}[Zvezno porazdeljene slučajne spremenljivke]

Slučajna spremenljivka $X$ je \textit{zvezno porazdleljena}, če obstaja \textit{nenegativna integrabilna} funkcija $p_x$, imenovana \textit{gostota verjetnosti}, da za $\forall x \in \mathbb{R}$ velja
$$F_X(x) = \int_{-\infty}^x p_x(t) ~dt.$$
Tedaj je $F_X = F$ \textit{zvezna} funckija, toda obstajajo zvezno porazdeljene funkcije, ki nimajo gostote (torej jih ni mogoče zapisati s tistim integralom). \\

\noindent Ker je $\mathlarger{\lim_{n \rightarrow \infty} F(x) = 1}$, je $\mathlarger{\int_{-\infty}^{\infty} p(t) = 1}$. Če je $p$ \textit{zvezna} v točki $x$, potem je $F$ \textit{odvedljiva} v $x$ in velja $F'(x) = p(x)$. \\

\noindent Za $\forall x \in \mathbb{R}$ velja $\mathbb{P}(X = x) = F(x) - F(x-) = 0$. Če je $x_1 < x_2$, potem je 
$$\mathbb{P}(x_1 \leq X \leq x_2) ~=~ F(x_2) - F(x_1-) ~=~ \int_{x_1}^{x_2} p(t) ~dt.$$ \\

\noindent Nekatere pomembnejše zvezne porazdelitve:
\begin{enumerate}
	\item \textsc{Enakomerna zvezna porazdelitev} na $[a, b]$:
	 $$p(x) ~=~ \begin{cases}
	 	\frac{1}{b-a} &; ~\text{če}~ a \leq x \leq b \\
	 	0 &; ~\text{sicer}
	 \end{cases}$$
	 $$F(x) ~=~ \int_{-\infty}^x p(t) ~dt ~=~ \begin{cases}
	 	0 &; ~\text{če}~ x \leq a \\
	 	\frac{x-a}{b-a} &; ~\text{če}~ a \leq x \leq b \\
	 	1 &; ~\text{če}~ x \geq b
	 \end{cases}$$
	 
	\item \textsc{Normalna ali Gaussova porazdelitev}: $N(\mu, \sigma), ~\mu \in \mathbb{R}, ~\sigma > 0$
	$$\mathlarger{p(x)~=~ \frac{1}{\sigma \sqrt{2 \pi}} \cdot e^{-\frac{1}{2} \left( \frac{x-\mu}{\sigma} \right)^2}}$$
V vseh primerih je ploščina pod grafom enaka $1$. \\

$N(0, 1)$: standardna normalna porazdelitev
$$\mathlarger{f(x) ~=~ \frac{1}{\pi \sqrt{2}} \cdot e^{-\frac{x^2}{2}}}$$
Laplaceova formula: za velike $n$ je $Bin(n, p) \approx N(np, \sqrt{npq})$

	\item \textsc{Eksponentna porazdelitev}: $Exp(\lambda), ~\lambda > 0$
	$$p(x) ~=~ \begin{cases}
		\lambda \cdot e^{-\lambda x} &; ~\text{če}~ x \geq 0 \\
		0 &; ~\text{sicer}~
	\end{cases}$$
	$$F(x) ~=~ \begin{cases}
		1 - e^{-\lambda x} &; ~\text{če}~ x \leq 0 \\
		0 &; ~\text{sicer}
	\end{cases}$$
	
	\item \textsc{Cauchyjeva porazdelitev}:
	$$p(x) ~=~ \frac{1}{\pi (1 + x^2)}, ~~~x \in \mathbb{R}$$
\end{enumerate}

\end{definicija}
\vspace{0.5cm}

\pagebreak

% #################################################################################################

\section{SLUČAJNI VEKTOR IN NEODVISNOST}
\vspace{0.5cm}

\begin{definicija}

\textit{Slučajni vektor} je $n$-terica slučajnih spremenljivk \\$X = (X_1, X_2, \ldots, X_n): \Omega \rightarrow \mathbb{R}^n$, z lastnostjo, da je množica \\$(X_1 \leq x_1, \ldots, X_n \leq x_n) := \{ \omega \in \Omega: X_1(\omega) \leq x_1, \ldots, X_n(\omega) \leq x_n \}$ dogodek za vsako $n$-terico $x = (x_1, x_2, \ldots, x_n) \in \mathbb{R}^n$.

\end{definicija}
\vspace{0.5cm}

\begin{definicija}

\textit{Porazdelitvena funkcija} $F_X: \mathbb{R}^n \rightarrow \mathbb{R}$ je definirana s predpisom $F_X(x) = F_{(X_1, \ldots, X_n)}(x_1, \ldots, x_n) = \mathbb{P}(X_1 \leq x_1, \ldots, X_n \leq x_n)$. Za $\forall x \in \mathbb{R}^n$ je $F(x) \in [0, 1]$; glede na vsako spremenljivko je $F$ naraščajoča in z desne zvezna;
$$\lim_{x_i \rightarrow \infty, ~\forall i} F(x_1, \ldots, x_n) ~=~ 1, ~~\lim_{x_i \rightarrow -\infty, ~\forall i} F(x_1, \ldots, x_n) ~=~ 0.$$
Če pošljemo proti $\infty$ samo nekatere spremenljivke, dobimo porazdelitveno funkcijo pod vektorjem, npr.
$$\lim_{x_n \rightarrow \infty} F(x_1, \ldots, x_n) ~=~ F_{(X_1, \ldots, X_{n-1})}(x_1, x_2, \ldots, x_{n-1}).$$
Robna porazdelitve funkcija $X_1$, \textit{robna (marginalna)} porazdelitev:
$$\lim_{\substack{x_2 \rightarrow \infty \\ \cdots \\ x_n \rightarrow \infty}} ~=~ F_{X_1}(x_1)$$

\end{definicija}
\vspace{0.5cm}

\begin{definicija}

Slučajne spremenljivke $X_1, \ldots, X_n$ v slučajnem vektorju $X = (X_1, \ldots, X_n)$ so \textit{neodvisne}, če za $\forall x_1, \ldots, x_n \in \mathbb{R}$ velja 
$$F_X(x_1, \ldots, x_n) = F_{X_1}(x_1) \cdot \ldots \cdot F_{X_n}(x_n).$$
Drugače:
\begin{align*}
\mathbb{P}(X_1 \leq x_1, \ldots, X_n \leq x_n) ~&=~ \mathbb{P}(X_1 \leq x_1) \cdot \ldots \cdot \mathbb{P}(X_n \leq x_n) \\ 
~&=~ (X_1 \leq x_1, \ldots, X_n \leq x_n) ~~\text{so neodvisne}.
\end{align*}

\end{definicija}
\vspace{0.5cm}

\begin{trditev}

Naj bo $(X, Y)$ diskreten slučajni vektor; \\$p_{i,j} = \mathbb{P}(X = x_i, Y = y_j)$, $p_i ~=~ \mathbb{P}(X = x_i), q_j ~=~ \mathbb{P}(Y = y_j)$, $i = 1, 2, \ldots$, $j = 1, 2, \ldots$.
$$X ~\text{in}~ Y ~\text{sta neodvisni slučajni spremenljivki}~ \Leftrightarrow ~p_{i,j} = p_i \cdot q_j, ~~\forall i, ~\forall j $$

\end{trditev}
\vspace{0.5cm}

% #################################################################################################

\section{MATEMATIČNO UPANJE \\oz. pričakovana vrednost}
\vspace{0.5cm}

\begin{definicija}

Za končno slučajno spremenljivko $X: ~\begin{pmatrix}
	x_1 & x_2 & \cdots & x_n \\
	p_1 & p_2 & \cdots & p_n
\end{pmatrix}$ je matematično upanje definirano kot 
$$E(x) ~=~ \sum_{k=1}^n x_k \cdot p_k.$$
Naj ima seda $X$ neskončno zalogo vrednosti. Če je $X$ \textit{diskretna} s $p_k = \mathbb{P}(X = x_k)$ ($k \in \mathbb{N}$), potem $X$ ima matematično upanje, če je 
$$\sum_{k=1}^{\infty} |x_k| \cdot p_k ~<~ \infty ~~~\text{(je končno)};$$
tedaj je matematično upanje definirano kot vsota vrste 
$$E(x) ~:=~ \sum_{k=1}^{\infty} x_k \cdot p_k.$$
Če je $X$ \textit{zvezna} z gostoto $p(x)$, potem rečemo, da $X$ ima matematično upanje, če je
$$\int_{-\infty}^{\infty} |x| \cdot p_k(x) ~dx ~<~ \infty;$$
tedaj je matematično upanje definirano kot
$$E(x) ~:=~ \int_{-\infty}^{\infty} x \cdot p(x) ~dx.$$

\end{definicija}
\vspace{0.5cm}

\begin{trditev}

Naj bo $f: \mathbb{R} \rightarrow \mathbb{R}$ funkcija.
\begin{enumerate}
	\item[a)] Če je $X: ~\begin{pmatrix}
	x_1 & x_2 & x_3 & \cdots \\
	p_1 & p_2 & p_3 & \cdots
\end{pmatrix}$, potem je 
	$$E(f \circ X) ~=~ \sum_k f(x_k) \cdot p_k,$$
	če matematično upanje obstaja, torej je vrsta \textit{absolutno kovnergentna}.
	\item[b)] Če je $X$ zvezno porazdeljena z gostoto $p(x)$, potem je 
	$$E(f \circ X) ~=~ \int_{-\infty}^{\infty} f(x) \cdot p(x) ~dx,$$
	če je integral \textit{absolutno konvergenten}.
\end{enumerate}

\end{trditev}
\vspace{0.5cm}

\begin{posledica}

Slučajna spremenljivka $X$ ima matematično upanje $\Leftrightarrow$ ko ga ima slučajna spremenljivka $(X)$. Tedaj velja
$$|E(x)| ~\leq~ E(|x|).$$

\end{posledica}
\vspace{0.5cm}

\begin{posledica}

Za $a \in \mathbb{R}$ in slučajno spremenljivko $X$ z matematičnim upanjem velja
$$E(a \cdot X) ~=~ a \cdot E(x).$$ 

\end{posledica}
\vspace{0.5cm}

\begin{trditev}

Naj bo $f: \mathbb{R}^2 \rightarrow \mathbb{R}$ funkcija, $(X, Y)$ diskretno porazdeljen slučajni vektor:
$$p_{ij} ~=~ \mathbb{P}(X = x_i, Y = y_i), ~~~i,j = 1, 2, \ldots$$
Potem je $f(X, Y): \Omega \rightarrow \mathbb{R}$ slučajna spremenljivka in velja
$$E(f(X, Y)) ~=~ \sum_i \sum_j f(x_i, y_j) \cdot p_{ij},$$
če le vrsta absolutno konvergira.

\end{trditev}
\vspace{0.5cm}

\begin{trditev}

Če imata $X$ in $Y$ matematično upanje, ga ima tudi $X + Y$ in velja
$$E(X + Y) ~=~ E(X) + E(Y).$$

\end{trditev}
\vspace{0.5cm}

\begin{posledica}

Za slučajne spremenljivke $X_1, X_2, \ldots, X_n$, ki imajo matematično upanje velja:
$$E(a_1 X_1 + a_2 X_2 + \ldots + a_n X_n) ~=~ a_1 \cdot E(X_1) + \ldots + a_n \cdot E(X_n),$$
kjer so $a_1, a_2, \ldots, a_n \in \mathbb{R}$.

\end{posledica}
\vspace{0.5cm}

\begin{trditev}

Če obstaja $E(X^2)$ in $E(Y^2)$, potem obstaja tudi $E(|XY|)$ in velja
$$E(|XY|) ~\leq~ \sqrt{E(X^2) \cdot E(Y^2)} ~~(\textit{Cauchy-Schwarzova neenakost})$$
Enakost velja $~\Leftrightarrow~$ $|Y| = \sqrt{\cfrac{E(Y^2)}{E(X^2)}} \cdot |X|$ z verjetnostjo $1$.

\end{trditev}
\vspace{0.5cm}

\begin{posledica}

Če obstaja $E(X^2)$, potem obstaja tudi $E(|X|)$ in velja
$$(E(|X|))^2 ~\leq~ E(X^2).$$

\end{posledica}
\vspace{0.5cm}

\begin{trditev}

Naj bosta $X$ in $Y$ \textit{neodvisni} slučajni spremenljivki, ki imata matematično upanje. Potem obstaja tudi matematično upanje $X \cdot Y$ in velja
$$E(X \cdot Y) ~=~ E(X) \cdot E(Y).$$

\end{trditev}
\vspace{0.5cm}

\pagebreak

% #################################################################################################

\section{DISPERZIJA, KOVARIANCA IN KORELACIJSKI KOEFICIENT}
\vspace{0.5cm}

\begin{definicija}

Naj obstaja $E(X^2)$. \textit{Disperzija} oz. \textit{varianca} je definirana kot
$$D(X) \equiv Var(X) := E((X - E(X))^2).$$
$D(X)$ meri razpršenost okoli $E(X)$. Velja
$$D(X) = E(X^2) - (E(X)^2)$$

\end{definicija}
\vspace{0.5cm}

\begin{trditev}[Lastnosti $D(X)$]
~\\
\begin{enumerate}
	\item $D(X) \geq 0$ \\
	$D(X) = 0 ~\Leftrightarrow~ \mathbb{P}(X = E(X)) = 1$
	\item $D(a \cdot X) = a^2 \cdot D(X), ~~a \in \mathbb{R}$
	\item Za $\forall a \in \mathbb{R}$ je 
	$$E((X - a)^2) \geq D(X)$$
	Enačaj velje $\Leftrightarrow$ $a = E(X)$.
\end{enumerate}

\end{trditev}
\vspace{0.5cm}

\begin{definicija}

\textit{Standardna deviacija} oz. \textit{standardni odklon} je definiran kot
$$\sigma(X) := \sqrt{D(X)}.$$
Zanjo velja
$$\sigma(a \cdot X) = |a| \cdot \sigma(X), ~~a \in \mathbb{R}.$$

\end{definicija}
\vspace{0.5cm}

\begin{trditev}[Nekatere $E(X)$ in $D(X)$]
~
\begin{enumerate}
	\item \textsc{Enakomerna porazdelitev}:
	$$E(X) = \frac{x_1 + \ldots + x_n}{n}$$
	$$D(X) = \frac{x_1^2 + \ldots + x_n^2}{n} - \left( \frac{x_1 + \ldots + x_n}{n} \right)^2$$
	\item \textsc{Binomska porazdelitev}, $Bin(n, p)$:
	$$E(X) = n \cdot p$$
	$$D(X) = n \cdot p \cdot q$$
	$$\sigma(X) = \sqrt{n \cdot p \cdot q}$$
	\item \textsc{Poissonova porazdelitev}, $Poi(\lambda)$:
	$$E(X) = D(X) = \lambda$$
	\item \textsc{Geometrijska porazdelitev}, $Geo(p)$:
	$$E(X) = \frac{1}{p}$$
	$$D(X) = \frac{q}{p^2}, ~~q = 1 - p$$
	\item \textsc{Pascalova porazdelitev}, $Pas(m, p)$:
	$$E(X) = \frac{m}{p}$$
	$$D(x) = \frac{m \cdot q}{p}$$
	\item \textsc{Enakomerna zvezna porazdelitev} na $[a, b]$:
	$$E(X) = \frac{a + b}{2}$$ $$
	D(X) = \frac{(b-a)^2}{12}$$
	\item \textsc{Normalna porazdelitev}, $N(\mu, \sigma)$, $\mu \in \mathbb{R}$, $\sigma > 0$:
	$$E(X) = \mu$$
	$$D(X) = \sigma^2$$
	$$\sigma(X) = \sigma$$
	\item \textsc{Eksponentna porazdelitev}, $Exp(\lambda)$:
	$$E(X) = \frac{1}{\lambda}$$
	$$D(X) = \frac{1}{\lambda^2}$$
\end{enumerate}

\end{trditev}
\vspace{0.5cm}

\begin{definicija}

\textit{Kovarianca} slučajnih spremenljivk $X$ in $Y$ je definirana kot 
\begin{align*}
K(X, Y) \equiv Cov(X, Y) :&= E((X - E(X)) \cdot (Y - E(Y))) \\
&= E(X \cdot Y) - E(X) \cdot E(Y).
\end{align*}


\end{definicija}
\vspace{0.5cm}

\begin{trditev}[Lastnostni kovariance]
~\\
\begin{enumerate}
	\item $K(X,X) = D(X$
	\item $X$ in $Y$ sta \textit{nekorelirana} $~\Leftrightarrow~$ $K(X, Y) = 0$
	\item $K$ je \textit{simetrična} in \textit{linearna}:
	$$K(Y, X) ~=~ K(X, Y)$$
	$$K(aX + bY, Z) ~=~ a \cdot K(X, Z) + b \cdot K(Y, Z), ~~\text{za}~~ a, b \in \mathbb{R}$$
	\item Kovarianca obstaja, če obstajata obe disperziji $D(X)$ in $D(Y)$. Tedaj velja
	$$|K(X,Y)| ~\leq~ \sqrt{D(X) \cdot D(Y)} ~=~ \sigma(x) \cdot \sigma(y)$$
	Enakost velja $~\Leftrightarrow~$ $Y - \mathbb{E}(Y) ~=~ \pm \frac{\sigma(Y)}{\sigma(X)} \cdot (X - \mathbb{E}(X))$ z verjetnostjo $1$
	\item Če $X$ in $Y$ imata disperzijo, potem jo ima tudi $X+Y$ in velja
	$$D(X+Y) ~=~ D(X) + D(Y) + 2K(X,Y)$$
	Če sta $X$ in $Y$ nekorelirani, potem je
	$$D(X+Y) ~=~ D(X) + D(Y)$$
	\item Posplošitev  zadnje lastnosti:
	$$D(X_1 + X_2 + \ldots + X_n) ~=~ \sum_{k=1}^n D(X_n) + 2\sum_{i=1}^{n-1} \sum_{j=i+1}^n K(X_i, X_j)$$
	Posebej: če so $X_1, \ldots, X_n$ paroma nekorelirane, potem je 
	$$D(X_1, \ldots, X_n) ~=~ D(X_1) + \ldots + D(X_n)$$
\end{enumerate}
\end{trditev}
\vspace{0.5cm}

\begin{definicija}

\textit{Standardizacija} slučajne spremenljivke $X$ je slučajna spremenljvka
$$X_s ~=~ \frac{X - \mathbb{E}(X)}{\sigma(X)}.$$
Tedaj je $\mathbb{E}(X_s) = 0$, $D(X_s) = 1$, saj je
$$D(X_s) ~=~ \frac{1}{\sigma(X)^2} \cdot D(X - \mathbb{E}(X)) ~=~ 1.$$

\end{definicija}
\vspace{0.5cm}

\begin{definicija}

\textit{Korelacijski koeficient} slučajnih spremenljivk $X$ in $Y$ je
$$r(X,Y) ~=~ \frac{K(X, Y)}{\sigma(X) \cdot \sigma(Y)} ~=~ \frac{\mathbb{E}((X - \mathbb{E}(X)) \cdot (Y - \mathbb{E}(Y)))}{\sigma(X) \cdot \sigma(Y)} ~=~ \mathbb{E}(X_s \cdot Y_s)$$

\end{definicija}
\vspace{0.5cm}

\begin{trditev}[Lastnosti korelacijskih koeficientov]
~\\
\begin{enumerate}
	\item $r(X,Y) = 0 ~\Leftrightarrow~$ $X$ in $Y$ sta nekorelirani
	\item $-1 \leq r(X,Y) \leq 1$ sledi iz 4. lastnosti kovariance
	\item $r(X,Y) = \pm 1$ $~\Leftrightarrow~$ $Y = \pm \frac{\sigma(Y)}{\sigma(X)} \cdot (X - \mathbb{E}(X)) + \mathbb{E}(Y)$ z verjetnostjo $1$
\end{enumerate}
\end{trditev}
\vspace{0.5cm}

\pagebreak

% #################################################################################################

\section{POGOJNA PORAZDELITEV IN POGOJNO MATEMATIČNO UPANJE}
\vspace{0.5cm}

\begin{definicija}

Fiksirajmo dogodek $B$ s $\mathbb{P}(B) > 0$. \textit{Pogojna porazdelitvena funkcija} slučajne spremenljivke $X$ glede na pogoj $B$ je
$$F_X(x | B) ~=~ \mathbb{P}(X \leq x | B) ~=~ \frac{\mathbb{P}((X \leq x) \cap B)}{\mathbb{P}(B)}$$
in ima enake lastnosti kot porazdelitvena funkcija. \\

\noindent Naj bo $(X, Y)$ diskreten slučajni vektor: 
$$p_{ij} =  \mathbb{P}(X = x_i, Y = y_j)$$
$$B := (Y = y_j), ~~\mathbb{P}(B) = \mathbb{P}(Y = y_j) = q_j$$
Potem je \textit{pogojna porazdelitvena funkcija} slučajne spremenljivke $X$ glede na $Y = y_j$:
\begin{align*}
	F_X(x | y_j) ~&=~ F_X(x | Y = y_j) ~=~ \mathbb{P}(X \leq y |Y = y_j) ~= \\
	&=~ \frac{1}{q_j} \mathbb{P}((X \leq x) \cap (Y = y_j)) ~=~ \frac{1}{q_j} \sum_{i: x_1 \leq x} p_{ij}
\end{align*}
Vpeljimo  \textit{pogojno verjetnostno funkcijo}:
$$p_{i|j} ~=~ \mathbb{P}(X = x_i | Y = y_j) ~=~ \frac{\mathbb{P}(X = x_i, Y = y_j)}{\mathbb{P}(Y = y_j)} ~=~ \frac{p_{ij}}{q_j}.$$
Tedaj je 
$$F_X(x | Y = y_j) ~=~ \sum_{i: x_i \leq x} p_{i|j}.$$

\end{definicija}
\vspace{0.5cm}

\begin{definicija}

\textit{Pogojno matematično upanje} slučajne spremenljivke $X$ glede na $Y = y_j$ je matematično upanje te porazdelitve:
$$\mathbb{E}(X | y_j) ~\equiv~ \mathbb{E}(X | Y = y_j) ~=~ \sum_i x_i \cdot p_{i|j} ~=~ \frac{1}{q_j} \sum_i x_i \cdot p_{ij}$$
Tako dobimo novo slučajno spremenljivko:
$$\mathbb{E}(X | Y): ~\begin{pmatrix}
	\mathbb{E}(X | y_1) & \mathbb{E}(X | y_2) & \ldots \\
	q_1 & q_2 & \ldots
\end{pmatrix}$$
Označimo $\varphi(y_j) := \mathbb{E}(X | y_j)$ za $\forall j$:
$$\mathbb{E}(X | Y) ~:=~ \varphi(Y): ~\begin{pmatrix}
	\varphi(y_1) & \varphi(y_2) & \ldots \\
	q_1 & q_2 & \ldots
\end{pmatrix}$$
$\varphi$ je \textit{regresijska funkcija}.

\end{definicija}
\vspace{0.5cm}

\pagebreak

% #################################################################################################

\section{RODOVNE FUNKICJE}
\vspace{0.5cm}

\begin{definicija}

Naj bo $X$ slučajna spremenljivka z vrednostmi v $\mathbb{N} \cup \{0\}$:
\begin{align*}
	p_k ~&=~ \mathbb{P}(X = k), ~~~~~k ~=~ 0, 1, 2, \ldots, \\
	p_k ~&\leq~ 0, ~~~~~~~~~~\sum_{k=0}^{\infty} p_k ~=~ 1.
\end{align*}
\textit{Rodovna funkcija} slučajne spremenljivke $X$ je 
$$G_X(s) ~:=~ p_0 + p_1 \cdot s + p_2 \cdot s^2 + p_3 \cdot s^3 + \ldots ~=~ \sum_{k=0}^{\infty} p_k \cdot s^k$$
za $\forall s \in \mathbb{R}$, za katere vrsta \textit{absolutno konvergira}. \\

\noindent Očitno je $G_X(0) = p_0$, $G_X(1) = \mathlarger{\sum_{k=0}^{\infty} p_k} = 1$, $G_X(s) = \mathbb{E}(s^X)$, saj je
$$s^X: ~\begin{pmatrix}
	~s^0 & s^1 & s^2 & s^3 & \ldots~ \\
	~p_0 & p_1 & p_2 & p_3 & \ldots~
\end{pmatrix}$$
Za $s \in [-1, 1]$ velja $|p_k \cdot s^k| \leq p_k$ in $\mathlarger{\sum_{k=0}^{\infty} = 1}$, zato vrsta $\mathlarger{\sum_{k=0}^{\infty}} |p_k \cdot s^k|$ konvergira. Torej je konvergenčni radij vrste vsaj $1$.

\end{definicija}
\vspace{0.5cm}

\begin{izrek}[Izrek o enoličnosti]

Naj imata $X$ in $Y$ rodovni funkciji $G_X$ in $G_Y$. Potem je
$$G_X(s) = G_Y(s) ~\text{za}~ \forall s \in [-1, 1] ~\Leftrightarrow~ \mathbb{P}(X=k) = \mathbb{P}(Y=k) ~\text{za}~ \forall k = 0, 1, 2, \ldots $$
Tedaj velja
$$\mathbb{P}(X = k) ~=~ \frac{1}{k!} \cdot G_X^{(k)}(0).$$

\end{izrek}
\vspace{0.5cm}

\begin{izrek}

Naj ima $X$ ima rodovno funkcijo $G_X$ in $n \in \mathbb{N}$. Potem je 
$$G_X^{(n)}(1-) ~=~ \mathbb{E}(X(X-1)(X-2)\ldots(X-n+1)),$$
kjer je 
$$G_X^{(n)}(1-) ~=~ \lim_{s \nearrow 1} G_X^{(n)}(s).$$

\end{izrek}
\vspace{0.5cm}

\begin{posledica}

$$\mathbb{E}(X) ~=~ G_X'(1-)$$
\begin{align*}
	D(X) ~&=~ \mathbb{E}(X(X-1)) + \mathbb{E}(X) - (\mathbb{E}(X))^2 = \\
	&=~ G_X''(1-) + G_X'(1-) - (G_X'(1-))^2
\end{align*}

\end{posledica}
\vspace{0.5cm}

\begin{izrek}

Naj bosta $X$ in $Y$ neodvisni slučajni spremenljivki z rodovnimi funkcijami $G_X$ in $G_Y$. Potem je 
$$G_{X+Y}(s) ~=~ G_X(s) \cdot G_Y(s) ~~\text{za}~ s \in [-1, 1].$$

\end{izrek}
\vspace{0.5cm}

\begin{posplošitev}

Če je $S_n = X_1 + \ldots + X_n$ vsota neodvisnih slučajnih spremenljivk z vrednostmi v $\mathbb{N} \cup \{0\}$, potem je za $s \in [-1, 1]$
$$G_{S_n}(s) ~=~ G_{X_1}(s) \cdot G_{X_2}(s) \cdot \ldots \cdot G_{X_n}(s).$$
Posebej: če so $X_1, X_2, \ldots, X_n$ enako porazdeljene, potem je
$$G_{S_n}(s) ~=~ (G_X(s))^n.$$ 

\end{posplošitev}
\vspace{0.5cm}

\begin{izrek}

Naj bodo za $\forall n \in \mathbb{N}$ slučajne spremenljivke $N, X_1, X_2, \ldots, X_n$ neodvisne. Naj ima $N$ rodovno funkcijo $G_N$, $X_n$ pa rodovno funkcijo $G_X$ za $\forall n \in \mathbb{N}$. Potem ima slučajna spremenljivka $S = X_1 + X_2 + \ldots + X_N$ rodovno funkcijo
$$G_S(s) ~=~ G_N(G_X(s)), ~~s \in [-1, 1].$$ 

\end{izrek}
\vspace{0.5cm}

\begin{posledica}[Waldova enakost]

$$\mathbb{E}(S) ~=~ \mathbb{E}(N) \cdot \mathbb{E}(X)$$

\end{posledica}
\vspace{0.5cm}

\pagebreak

% #################################################################################################

\section{VIŠJI MOMENTI in VRSTILNE \\KARAKTERISTIKE}
\vspace{0.5cm}

\begin{definicija}

Naj bo $k \in \mathbb{N}$ in $a \in \mathbb{R}$. \textit{Moment reda k glede na a} je
$$m_k(a) ~=~ \mathbb{E}((X-a)^k),$$
če obstaja. Za $a$ občajno vzamemo:
\begin{enumerate}
	\item \textit{začetni moment}: \\$a = 0$: $$z_k = m_k(0) = \mathbb{E}(X^k)$$
	\item \textit{centralni moment reda $k$}: \\$a = \mathbb{E}(X)$: $$m_k = m_k(\mathbb{E}(X)) = \mathbb{E}((X - \mathbb{E}(X))^k)$$
\end{enumerate}
Očitno je $z_1 = \mathbb{E}(X)$ in $m_2 = D(X)$.

\end{definicija}
\vspace{0.5cm}

\begin{trditev}

Če obstaja $m_n(a)$, potem obstaja tudi $m_k(a)$ za $\forall k < n$.

\end{trditev}
\vspace{0.5cm}

\begin{trditev}

Če obstaja začetni moment $z_n$, potem obstaja tudi $m_n(a)$ za $\forall a \in \mathbb{R}$.

\end{trditev}
\vspace{0.5cm}

\begin{definicija}

\textit{Asimetrija} slučajne spremenljivke $X$ je 
$$A(X) ~:=~ \mathbb{E}({X_s}^3) ~=~ \mathbb{E}\left( \left( \frac{X - \mathbb{E}(X)}{\sigma(X)} \right)^3 \right) ~=~ \frac{m_3}{(m_2)^{\frac{3}{2}}}$$
Velja:
\begin{itemize}
	\item $A(N(\mu, \sigma)) = 0$
	\item $A(\lambda X) = A(X)$ za $\lambda > 0$.
\end{itemize}

\end{definicija}
\vspace{0.5cm}

\begin{definicija}

\textit{Sploščenost (kurtozis)} slučajne spremenljivke $X$ je
$$K(X) ~=~ \frac{m_4}{(m_2)^2} ~=~ \mathbb{E}({X_s}^4) ~=~ \mathbb{E}\left( \left( \frac{X - \mathbb{E}(X)}{\sigma(X)} \right)^4 \right)$$
Velja:
\begin{itemize}
	\item $K(N(\mu, \sigma)) = 3$
	\item $K(\lambda X) = K(X)$ za $\lambda > 0$.
\end{itemize}

\end{definicija}
\vspace{0.5cm}

\begin{opomba}

Nekateri definirajo sploščenost kot $K(X) - 3$, torej je v primeru $N(\mu, \sigma)$ enako $0$.

\end{opomba}
\vspace{0.5cm}

\begin{definicija}

\textit{Mediana}  slučajne spremenljivke $X$ je vsaka vrednost $x \in \mathbb{R}$, za katere velja
$$\mathbb{P}(X \leq x) ~\geq~ \frac{1}{2} ~~~\text{in}~~~ \mathbb{P}(X \geq x) ~\geq~ \frac{1}{2}.$$
Ker je 
$$\mathbb{P}(X \geq x) ~=~ 1 - \mathbb{P}(X < x) ~=~ 1 - F(x-),$$
lahko pogoj za mediano zapišemo takole
$$F(x-) ~\leq~ \frac{1}{2} ~\leq~ F(x).$$
Če je $X$ zvezno porazdeljena, je pogoj enak $F(x) = \cfrac{1}{2}$. Te vrednosti označimo z $x_{\frac{1}{2}}$.

\end{definicija}
\vspace{0.5cm}

\begin{definicija}

\textit{Kvantil reda $p$} je vsaka vrednost $x_p$, za katero velja
$$\mathbb{P}(X \leq x_p) ~\geq~ p ~~~\text{in}~~~ \mathbb{P}(X \geq x_p) ~\geq~ 1 - p$$
oziroma ekvivalentno:
$$F(x_p-) ~\leq~ p ~\leq~ F(x_p), ~~0 < p < 1.$$
Če je $X$ zvezno porazdeljena, je pogoj za kvantil:
$$F(x_p) ~=~ p ~~~\text{oziroma}~~~ \int_{-\infty}^{x_p} p_X(t) dt ~=~ p.$$
\textit{(Semiinter)kvartilni razmik} je
$$s ~=~ \frac{1}{2}(x_{\frac{3}{4}} - x_{\frac{1}{4}})$$
in je nadomestek za standardno deviacijo.

\end{definicija}
\vspace{0.5cm}

% #################################################################################################

\section{MOMENTNO RODOVNE FUNKCIJE}
\vspace{0.5cm}

\begin{definicija}

$$M_X(t) ~=~ \mathbb{E}(e^{tX})$$
az $t \in \mathbb{R}$ za katere obstaja matematično upanje, torej $\mathbb{E}(e^{tX}) < \infty$. \\

\noindent Kadar ima $X$ vrednosti $\mathbb{N} \cup \{0\}$, je
$$M_X(t) ~=~ \mathbb{E}((e^t)^X) ~=~ G_X(e^t),$$
torej gre za posplošitev rodovne funkcije. \\

\noindent Če je $X$ \textit{zvezno porazdeljeno} z gostoto $p(x)$, je
$$M_X(t) ~=~ \int_{-\infty}^{\infty} e^{tx} p(x) dx. ~~~\text{(Laplaceova transformacija funkcije $p$)}$$

\end{definicija}
\vspace{0.5cm}

\begin{izrek}

Naj obstaja $\delta > 0$, da je $M_X(t) < \infty$ za $\forall \in (-\delta, delta)$. Potem je porazdelitev za $X$ natanko določena z $M_X$, vsi začetni momenti obstajajo,
$$z_k ~=~ \mathbb{E}(X^k) ~=~ {M_X}^{(k)}(0)$$
za $\forall k \in \mathbb{N}$ in
$$M_X(t) ~=~ \sum_{k=0}^{\infty} \frac{z_k}{k!} t^k$$
za $\forall t \in (-\delta, \delta)$.

\end{izrek}
\vspace{0.5cm}

% #################################################################################################


\end{document}
