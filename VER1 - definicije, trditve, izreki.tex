\documentclass[11pt]{article}
\usepackage[utf8]{inputenc}
\usepackage[slovene]{babel}

\usepackage{amsthm}
\usepackage{amsmath, amssymb, amsfonts}

\theoremstyle{definition}
\newtheorem{definicija}{Definicija}[section]

\theoremstyle{definition}
\newtheorem{trditev}{Trditev}[section]

\theoremstyle{definition}
\newtheorem{izrek}{Izrek}[section]

\theoremstyle{theorem}
\newtheorem*{posledica}{Posledica}

\theoremstyle{theorem}
\newtheorem*{opomba}{Opomba}

\newtheorem{lema}{Lema}
\newtheorem*{dokaz}{Dokaz}

\title{Verjetnost 1 - definicije, trditve in izreki}
\author{Oskar Vavtar}
\date{2020/21}

\begin{document}
\maketitle
\pagebreak
\tableofcontents
\pagebreak

% #################################################################################################

\section{DEFINICIJA VERJETNOSTI}

% *************************************************************************************************

\subsection{Neformalni uvod v verjetnost}

\begin{definicija}[Verjetnost]

Izvajamo poskus. Opazujemo določen pojav, ki ga imenujemo \textit{dogodek}. Poskus ponovimo $n$ - krat. \\
Definirajmo \textit{frekvenco dogodka} $k_n(A)$ kot število ponovitev, pri katerih se dogodek zgodi. \\
\textit{Relativna frekvenca} je definirana kot $f_n(A) = \frac{k_n(A)}{n}$.
Zaporedje $\{f_n(A)\}_{n \in \mathbb{N}}$ konvergira k nekem številu $p \in [0, 1].$ \\
\textsc{Statistična definicija verjetnosti} je definirana kot
$$\mathbb{P}(A) = \lim_{n\to\infty} \{f_n(A)\}.$$
\textsc{Klasična definicija verjetnosti} je definirana kot
$$\mathbb{P}(A) = \frac{\text{št. ugodnih izidov A}}{\text{št. vseh izidov A}}.$$
Če je izidov neskončno mnogo uporabimo geometrijsko definicijo verjetnosti.

\end{definicija}
\vspace{0.5cm}

% *************************************************************************************************

\subsection{Aksiomična definicija verjetnosti}
\vspace{0.5cm}

\begin{definicija}

Imamo \textit{prostor vseh izidov} oz. \textit{vzorčni prostor} $\Omega$. \textit{Dogodki} so nekatere (ne nujno vse) podmnožice $\Omega$.

\end{definicija}
\vspace{0.5cm}

\begin{definicija}[Operacije na dogodkih]
~
\begin{enumerate}
	\item \textsc{Vsota} oz. \textsc{unija} dogodkov:
	$$A + B = A \cup B$$
	je dogodek, ki se zgodi, če se zgodi vsaj eden od dogodkov $A$ in $B$. \\
	\item \textsc{Produkt} oz. \textsc{presek} dogodkov:
	$$A \cdot B = A \cap B$$
	je dogodek, ki se zgodi, če se zgodita oba dogodka $A$ in $B$ hkrati.
	\item \textsc{Nasprotni} dogodek oz. \textsc{komplement} dogodkov:
	$$\bar{A} = A^C$$
	je dogodek, ki se zgodi, če se dogodek $A$ ne zgodi.
\end{enumerate}
\end{definicija}
\vspace{0.5cm}

\begin{opomba}

Pravila za računanje z dogodki:
\begin{enumerate}
	\item Idempotentnost: 
	$$A \cup A = A = A \cap A$$
	\item Komutativnost:
	$$A \cup B = B \cup A$$
	$$A \cap B = B \cap A$$
	\item Asociativnost:
	$$(A \cup B) \cup C = A \cup (B \cup C)$$
	$$(A \cap B) \cap C = A \cap (B \cap C)$$
	\item Distributivnost:
	$$(A \cup B) \cap C = (A \cap C) \cup (B \cap C)$$
	$$(A \cap B) \cup C = (A \cup C) \cap (B \cup C)$$
	\item de Morganova zakona:
	$$(A \cap B)^C = A^C \cup B^C$$
	$$(A \cup B)^C = A^C \cap B^C$$
	Še več:
	$$\left( \bigcup_i A_i \right)^C = \bigcap_i {A_i}^C$$
	$$\left( \bigcap_i A_i \right)^C = \bigcup_i {A_i}^C$$
\end{enumerate}
\end{opomba}
\vspace{0.5cm}

\begin{opomba}

V splošnem ni vsaka podmnožica $A \subset \Omega$ dogodek.

\end{opomba}
\vspace{0.5cm}

\begin{definicija}[$\sigma$-algebra]

\textit{Neprazna} družina podmnožic (dogodkov) $\mathcal{F}$ v $\Omega$ je $\sigma$-algebra, če velja:
\begin{enumerate}
	\item \textit{Zaprtost komponente}:
	$$A \in \mathcal{F} \Rightarrow A^C \in \mathcal{F}$$
	\item \textit{Zaprtost števne unije}:
	$$A_1, A_2, A_3, \ldots \in \mathcal{F} \Rightarrow \bigcup_{i=1}^{\infty} A_i \in \mathcal{F}$$
\end{enumerate}
\end{definicija}
\vspace{0.5cm}

\begin{opomba}

Če v (2) zahtevamo manj:
$$A, B \in \mathcal{F} \Rightarrow A \cup B \in \mathcal{F},$$
je $\mathcal{F}$ \textit{algebra}. V algebri imamo torej zaprtost za končne unije in končne preseke, medtem ko je $\sigma$-algebra zaprta celo za števne preseke.

\end{opomba}
\vspace{0.5cm}

\begin{definicija}
Naj bo $\mathcal{F}$ $\sigma$-algebra, $\Omega$ vzorčni prostor. \textit{Verjetnostna mera} na $(\mathcal{F}, \Omega$), je preslikava $\mathbb{P}: \mathcal{F} \rightarrow \mathbb{R}$ za katero velja:
\begin{enumerate}
	\item $\mathbb{P} \geq 0$, za $\forall A \in \mathbb{F}$
	\item $\mathbb{P}(\Omega) = 1$
	\item Za poljubne paroma nerazdružljive dogodke velja
	$$\mathbb{P}\left( \bigcup_{i=1}^{\infty} A_i \right) = \sum_{n=1}^{\infty} \mathbb{P}(A_i)$$
\end{enumerate}
Trojico $(\mathcal{F}, \Omega, \mathbb{P})$ imenujemo \textit{verjetnostni prostor}.
\end{definicija}
\vspace{0.5cm}

\begin{posledica}[Posledice verjetnostnih aksiomov]
~
\begin{enumerate}

\item[(a)] $\mathbb{P}(\emptyset) = 0$ \\
\textbf{Dokaz:} v (\textit{3.}) vzamemo $A_i = \emptyset$: 
$\mathbb{P}(\emptyset) = \mathbb{P}(\emptyset) + \mathbb{P}(\emptyset) + \mathbb{P}(\emptyset) + \ldots$

\item[(b)] $\mathbb{P}$ je končno aditivna, torej za končno mnogo paroma nerazdružljivih dogodkov velja: \\
$\mathbb{P}(A_1 \cup \ldots \cup A_n) = \mathbb{P}(A_1) + \ldots + \mathbb{P}(A_n)$ \\
\textbf{Dokaz:} v (\textit{3.}) vzamemo $A_{n+1} = A_{n+2} = \ldots = \emptyset$ in uporabimo (\textit{a})

\item[(c)] $\mathbb{P}$ je \textit{monotona}, torej $A \subseteq B \Rightarrow \mathbb{P}(A) \subseteq \mathbb{P}(B)$ \\
Še več: $A \subseteq B \Rightarrow \mathbb{P}(B \setminus A) = \mathbb{P}(B) - \mathbb{P}(A)$ \\
\textbf{Dokaz:} ker je $B = A \cup (B-A)$, $A \cap (B \setminus A) = \emptyset$, zaradi (\textit{b}) velja $\mathbb{P}(B) = \mathbb{P}(A) + \mathbb{P}(B-A)$

\item[(d)] $\mathbb{P}(A^C) = 1 - \mathbb{P}(A)$ \\
\textbf{Dokaz:} v (\textit{c}) vzamemo $B = \Omega$

\item[(e)] $\mathbb{P}$ je zvezna:
\begin{enumerate}
	\item[(i)] $A_1 \subseteq A_2 \subseteq A_3 \subseteq \ldots \Rightarrow \mathbb{P}(\bigcup_{i=1}^{\infty} A_i) = \lim_{n \rightarrow \infty} \mathbb{P}(A_n)$
	\item[(ii)] $B_1 \supseteq B_2 \supseteq B_3 \supseteq \ldots \Rightarrow \mathbb{P}(\bigcap_{i=1}^{\infty} B_i) = \lim_{n \rightarrow \infty} \mathbb{P}(B_n)$
\end{enumerate} 
\textbf{Dokaz:}
\begin{enumerate}
	\item[(i)] Definiramo: $C_i = A_i \setminus A_{i-1}$ za $i = 2, 3, \ldots$, $C_1 = A_1$ \\
	Potem je $A_n = C_1 \cup \ldots \cup C_n$, $~C_i \cap C_j = \emptyset$ za $i \neq j$, \\ 
	$\bigcup_{i=1}^{\infty} A_i = \bigcup_{i=1}^{\infty} C_i$ \\
	Torej je 
	\begin{align*}
	\mathbb{P}(\bigcup_{i=1}^{\infty} A_i &= \mathbb{P}(\bigcup_{i=1}^{\infty} C_i) = \sum_{i=1}^{\infty} \mathbb{P}(C_i) = \\
	&= \lim_{n \rightarrow \infty} \sum_{i=1}^{n} \mathbb{P}(C_i) = \lim_{n \rightarrow \infty} \mathbb{P}(\bigcup_{i=1}^{n} C_i) = \lim_{n \rightarrow \infty} \mathbb{P}(A_n) \\
	\end{align*}
	\item[(ii)] Ker je $B_1 \supseteq B_2 \supseteq B_3 \supseteq \ldots$, sledi ${B_1}^C \subseteq {B_2}^C \subseteq {B_3}^C \subseteq \ldots$ \\
	Po (\textit{i}) je $\mathbb{P}(\bigcup_{i=1}^{\infty} {B_i}^C) = \lim_{i \rightarrow \infty} \mathbb{P}({D_i}^C)$, toda $\bigcup_{i=1}^{\infty} {B_i}^C = (\bigcap_{i=1}^{\infty} B_i)^C$. \\
	Zato je $1 - \mathbb{P}(\bigcap_{i=1}^{\infty} B_i) = \lim_{i \rightarrow \infty} (1 - \mathbb{P}(B_i))$, od koder sledi želena neenakost.
\end{enumerate}
\end{enumerate}
\end{posledica}
\vspace{0.5cm}

\end{document}
