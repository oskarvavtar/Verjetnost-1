\documentclass[11pt]{article}
\usepackage[utf8]{inputenc}
\usepackage[slovene]{babel}
\usepackage{relsize}
\usepackage{amsthm}
\usepackage{amsmath, amssymb, amsfonts}

\theoremstyle{definition}
\newtheorem{definicija}{Definicija}[section]

\theoremstyle{definition}
\newtheorem{trditev}{Trditev}[section]

\theoremstyle{definition}
\newtheorem{izrek}{Izrek}[section]

\newtheorem*{posledica}{Posledica}
\newtheorem*{opomba}{Opomba}
\newtheorem{lema}{Lema}
\newtheorem*{dokaz}{Dokaz}

\title{Verjetnost 1 - definicije, trditve in izreki}
\author{Oskar Vavtar}
\date{2020/21}

\begin{document}
\maketitle
\pagebreak
\tableofcontents
\pagebreak

% #################################################################################################

\section{DEFINICIJA VERJETNOSTI}

% *************************************************************************************************

\subsection{Neformalni uvod v verjetnost}

\begin{definicija}[Verjetnost]

Izvajamo poskus. Opazujemo določen pojav, ki ga imenujemo \textit{dogodek}. Poskus ponovimo $n$ - krat. \\
Definirajmo \textit{frekvenco dogodka} $k_n(A)$ kot število ponovitev, pri katerih se dogodek zgodi. \\
\textit{Relativna frekvenca} je definirana kot $f_n(A) = \frac{k_n(A)}{n}$.
Zaporedje $\{f_n(A)\}_{n \in \mathbb{N}}$ konvergira k nekem številu $p \in [0, 1].$ \\
\textsc{Statistična definicija verjetnosti} je definirana kot
$$\mathbb{P}(A) = \lim_{n\to\infty} \{f_n(A)\}.$$
\textsc{Klasična definicija verjetnosti} je definirana kot
$$\mathbb{P}(A) = \frac{\text{št. ugodnih izidov A}}{\text{št. vseh izidov A}}.$$
Če je izidov neskončno mnogo uporabimo geometrijsko definicijo verjetnosti.

\end{definicija}
\vspace{0.5cm}

% *************************************************************************************************

\subsection{Aksiomična definicija verjetnosti}
\vspace{0.5cm}

\begin{definicija}

Imamo \textit{prostor vseh izidov} oz. \textit{vzorčni prostor} $\Omega$. \textit{Dogodki} so nekatere (ne nujno vse) podmnožice $\Omega$.

\end{definicija}
\vspace{0.5cm}

\begin{definicija}[Operacije na dogodkih]
~
\begin{enumerate}
	\item \textsc{Vsota} oz. \textsc{unija} dogodkov:
	$$A + B = A \cup B$$
	je dogodek, ki se zgodi, če se zgodi vsaj eden od dogodkov $A$ in $B$. \\
	\item \textsc{Produkt} oz. \textsc{presek} dogodkov:
	$$A \cdot B = A \cap B$$
	je dogodek, ki se zgodi, če se zgodita oba dogodka $A$ in $B$ hkrati.
	\item \textsc{Nasprotni} dogodek oz. \textsc{komplement} dogodkov:
	$$\bar{A} = A^C$$
	je dogodek, ki se zgodi, če se dogodek $A$ ne zgodi.
\end{enumerate}
\end{definicija}
\vspace{0.5cm}

\begin{opomba}

Pravila za računanje z dogodki:
\begin{enumerate}
	\item Idempotentnost: 
	$$A \cup A = A = A \cap A$$
	\item Komutativnost:
	$$A \cup B = B \cup A$$
	$$A \cap B = B \cap A$$
	\item Asociativnost:
	$$(A \cup B) \cup C = A \cup (B \cup C)$$
	$$(A \cap B) \cap C = A \cap (B \cap C)$$
	\item Distributivnost:
	$$(A \cup B) \cap C = (A \cap C) \cup (B \cap C)$$
	$$(A \cap B) \cup C = (A \cup C) \cap (B \cup C)$$
	\item de Morganova zakona:
	$$(A \cap B)^C = A^C \cup B^C$$
	$$(A \cup B)^C = A^C \cap B^C$$
	Še več:
	$$\left( \bigcup_i A_i \right)^C = \bigcap_i {A_i}^C$$
	$$\left( \bigcap_i A_i \right)^C = \bigcup_i {A_i}^C$$
\end{enumerate}
\end{opomba}
\vspace{0.5cm}

\begin{opomba}

V splošnem ni vsaka podmnožica $A \subset \Omega$ dogodek.

\end{opomba}
\vspace{0.5cm}

\begin{definicija}[$\sigma$-algebra]

\textit{Neprazna} družina podmnožic (dogodkov) $\mathcal{F}$ v $\Omega$ je $\sigma$-algebra, če velja:
\begin{enumerate}
	\item \textit{Zaprtost komponente}:
	$$A \in \mathcal{F} \Rightarrow A^C \in \mathcal{F}$$
	\item \textit{Zaprtost števne unije}:
	$$A_1, A_2, A_3, \ldots \in \mathcal{F} \Rightarrow \bigcup_{i=1}^{\infty} A_i \in \mathcal{F}$$
\end{enumerate}
\end{definicija}
\vspace{0.5cm}

\begin{opomba}

Če v (2) zahtevamo manj:
$$A, B \in \mathcal{F} \Rightarrow A \cup B \in \mathcal{F},$$
je $\mathcal{F}$ \textit{algebra}. V algebri imamo torej zaprtost za končne unije in končne preseke, medtem ko je $\sigma$-algebra zaprta celo za števne preseke.

\end{opomba}
\vspace{0.5cm}

\begin{definicija}
Naj bo $\mathcal{F}$ $\sigma$-algebra, $\Omega$ vzorčni prostor. \textit{Verjetnostna mera} na $(\mathcal{F}, \Omega$), je preslikava $\mathbb{P}: \mathcal{F} \rightarrow \mathbb{R}$ za katero velja:
\begin{enumerate}
	\item $\mathbb{P} \geq 0$, za $\forall A \in \mathbb{F}$
	\item $\mathbb{P}(\Omega) = 1$
	\item Za poljubne paroma nerazdružljive dogodke velja
	$$\mathbb{P}\left( \bigcup_{i=1}^{\infty} A_i \right) = \sum_{n=1}^{\infty} \mathbb{P}(A_i)$$
\end{enumerate}
Trojico $(\mathcal{F}, \Omega, \mathbb{P})$ imenujemo \textit{verjetnostni prostor}.
\end{definicija}
\vspace{0.5cm}

\begin{posledica}[Posledice verjetnostnih aksiomov]
~
\begin{enumerate}

\item[(a)] $\mathbb{P}(\emptyset) = 0$ \\
\textbf{Dokaz:} v (\textit{3.}) vzamemo $A_i = \emptyset$: 
$\mathbb{P}(\emptyset) = \mathbb{P}(\emptyset) + \mathbb{P}(\emptyset) + \mathbb{P}(\emptyset) + \ldots$

\item[(b)] $\mathbb{P}$ je končno aditivna, torej za končno mnogo paroma nerazdružljivih dogodkov velja: \\
$\mathbb{P}(A_1 \cup \ldots \cup A_n) = \mathbb{P}(A_1) + \ldots + \mathbb{P}(A_n)$ \\
\textbf{Dokaz:} v (\textit{3.}) vzamemo $A_{n+1} = A_{n+2} = \ldots = \emptyset$ in uporabimo (\textit{a})

\item[(c)] $\mathbb{P}$ je \textit{monotona}, torej $A \subseteq B \Rightarrow \mathbb{P}(A) \subseteq \mathbb{P}(B)$ \\
Še več: $A \subseteq B \Rightarrow \mathbb{P}(B \setminus A) = \mathbb{P}(B) - \mathbb{P}(A)$ \\
\textbf{Dokaz:} ker je $B = A \cup (B-A)$, $A \cap (B \setminus A) = \emptyset$, zaradi (\textit{b}) velja $\mathbb{P}(B) = \mathbb{P}(A) + \mathbb{P}(B-A)$

\item[(d)] $\mathbb{P}(A^C) = 1 - \mathbb{P}(A)$ \\
\textbf{Dokaz:} v (\textit{c}) vzamemo $B = \Omega$

\item[(e)] $\mathbb{P}$ je zvezna:
\begin{enumerate}
	\item[(i)] $A_1 \subseteq A_2 \subseteq A_3 \subseteq \ldots \Rightarrow \mathbb{P}(\mathlarger{\bigcup_{i=1}^{\infty} A_i}) = \lim_{n \rightarrow \infty} \mathbb{P}(A_n)$
	\item[(ii)] $B_1 \supseteq B_2 \supseteq B_3 \supseteq \ldots \Rightarrow \mathbb{P}(\mathlarger{\bigcap_{i=1}^{\infty} B_i}) = \lim_{n \rightarrow \infty} \mathbb{P}(B_n)$
\end{enumerate} 
\textbf{Dokaz:}
\begin{enumerate}
	\item[(i)] Definiramo: $C_i = A_i \setminus A_{i-1}$ za $i = 2, 3, \ldots$, $C_1 = A_1$ \\
	Potem je $A_n = C_1 \cup \ldots \cup C_n$, $~C_i \cap C_j = \emptyset$ za $i \neq j$, \\ 
	$\mathlarger{\bigcup_{i=1}^{\infty} A_i} = \mathlarger{\bigcup_{i=1}^{\infty} C_i}$ \\
	Torej je 
	\begin{align*}
	\mathbb{P}(\bigcup_{i=1}^{\infty} A_i = \mathbb{P}(\bigcup_{i=1}^{\infty} C_i) &= \sum_{i=1}^{\infty} \mathbb{P}(C_i) = \\
	= \lim_{n \rightarrow \infty} \sum_{i=1}^{n} \mathbb{P}(C_i) &= \lim_{n \rightarrow \infty} \mathbb{P}(\bigcup_{i=1}^{n} C_i) = \lim_{n \rightarrow \infty} \mathbb{P}(A_n) \\
	\end{align*}
	\item[(ii)] Ker je $B_1 \supseteq B_2 \supseteq B_3 \supseteq \ldots$, sledi ${B_1}^C \subseteq {B_2}^C \subseteq {B_3}^C \subseteq \ldots$ \\
	Po (\textit{i}) je $\mathbb{P}(\bigcup_{i=1}^{\infty} {B_i}^C) = \mathlarger{\lim_{i \rightarrow \infty} \mathbb{P}({D_i}^C)}$, toda $\mathlarger{\bigcup_{i=1}^{\infty} {B_i}^C = \left( \bigcap_{i=1}^{\infty} B_i \right)^C}$. \\
	Zato je $1 - \mathbb{P}(\bigcap_{i=1}^{\infty} B_i) = \mathlarger{\lim_{i \rightarrow \infty}} (1 - \mathbb{P}(B_i))$, od koder sledi želena neenakost.
\end{enumerate}
\end{enumerate}
\end{posledica}
\vspace{0.5cm}

% *************************************************************************************************

\pagebreak

% #################################################################################################

\section{POGOJNA VERJETNOST}
\vspace{0.5cm}

\begin{definicija}[Pogojna verjetnost]

\textit{Pogojna verjetnost} dogodka $A$ glede na dogodek $B$, $\mathbb{P}(A | B)$, je verjetnost dogodka $A$ če vemo, da se je zgodil dogodek $B$.
Posplošimo:
$$\mathbb{P}(A | B) = \frac{\mathbb{P}(A \cap B)}{\mathbb{P}(B)}$$

\end{definicija}
\vspace{0.5cm}

\begin{posledica}

Iz definicije sledi
$$\mathbb{P}(A \cap B) = \mathbb{P}(A | B) \cdot \mathbb{P}(B).$$
Če posplošimo na $n$ dogodkov dobimo
$$\mathbb{P}(A_1 \cap A_2 \cap \ldots \cap A_n) = \mathbb{P}(A_1) \cdot \mathbb{P}(A_2 | A_1) \cdot \ldots \cdot \mathbb{P}(A_n | A_1 \cap \ldots \cap A_{n-1}).$$
Če velja $\mathbb{P}(A) = \mathbb{P}(A | B)$, sta dogodka neodvisna.

\end{posledica}
\vspace{0.5cm}

\begin{izrek}[Izrek o popolni verjetnosti]

Naj bo $(H_i)_i$ popoln sistem dogodkov. Potem je 
$$A = A \cap \Omega = A \cap {\left( \bigcup_i H_i \right)} = \bigcup_i A \cap H_i$$
in iz tega sledi
$$\mathbb{P}(A) = \sum_i \mathbb{P}(A \cap H_i) = \sum_i \mathbb{P}(H_i) \cdot \mathbb{P}(A | H_i).$$
To je formula za \textit{popolno verjetnost}.

\end{izrek}
\vspace{0.5cm}

\begin{posledica}[Bayesova formula]

Iz definicije pogojne verjetnosti vemo
$$\mathbb{P}(H_i | A) = \frac{\mathbb{P}(H_i \cap A}{\mathbb{P}(A)} = \frac{\mathbb{P}(H_i) \cdot \mathbb{P}(A | H_i)}{\mathbb{P}(A)}.$$
Če v imenovalec vstavimo izrek o popolni verjetnosti, dobimo \textit{Bayesovo formulo}:
$$\mathbb{P}(H_i | A) = \frac{\mathbb{P}(H_i) \cdot \mathbb{P}(A | H_i)}{\mathbb{P}(H_1) \cdot \mathbb{P}(A | H_1) + \ldots + \mathbb{P}(H_n) \cdot \mathbb{P}(A | H_n)} = \frac{\mathbb{P}(H_i) \cdot \mathbb{P}(A | H_i)}{\sum_i \mathbb{P}(H_i) \cdot \mathbb{P}(A | H_i)}$$

\end{posledica}
\vspace{0.5cm}

\begin{definicija}[Neodvisnost $2$ dogodkov]

Dogodka $A$ in $B$ sta \textit{neodvisna}, če je 
$$\mathbb{P}(A \cap B) = \mathbb{P}(A) \cdot \mathbb{P}(B).$$
Če je $\mathbb{P}(B) > 0$ to enakost lahko zapišemo kot
$$\mathbb{P}(A) = \frac{\mathbb{P}(A \cap B)}{\mathbb{P}(B)} = \mathbb{P}(A | B).$$

\end{definicija}
\vspace{0.5cm}

\begin{definicija}[Neodvisnost $k$ dogodkov]

Dogodki $(A_i)_i$ so \textit{neodvisni}, če za poljuben končen nabor različnih dogodkov $A_{i_1}, A_{i_2}, \ldots, A_{i_k}$ velja
$$\mathbb{P}(A_{i_1} \cap A_{i_2} \cap \ldots \cap A_{i_k}) = \mathbb{P}(A_{i_1}) \cdot \mathbb{P}(A_{i_2}) \cdot \ldots \cdot \mathbb{P}(A_{i_k}).$$
Če zahtevamo le za $k = 2$, torej $A_i$ in $A_j$ sta neodvisna le za $i \neq j$, potem rečemo, da so dogodki \textit{paroma neodvisni}. To je šibkejši pogoj kot neodvisnost.

\end{definicija}

\begin{trditev}

Če sta dogodka $A$ in $B$ neodvisna, potem sta neodvisna tudi dogodka $A^C$ in $B$, $A$ in $B^C$ ter $A^C$ in $B^C$.

\end{trditev}
\vspace{0.5cm}

% *************************************************************************************************

\pagebreak

% #################################################################################################

\section{ZAPOREDJA NEODVISNIH PONOVITEV \\POSKUSA}
\vspace{0.5cm}

\begin{definicija}[Bernoullijeva formula]

Imejmo zaporedje $n$ neodvisnih ponovitev poskusa, določenega z verjetnostnim prostorom $(\Omega, \mathcal{F}, p)$, v katerem je možen dogodek $A$ s \\$\mathbb{P}(A) = p$.
Z $A_n(k)$ označimo dogodek, da se $A$ zgodi natanko $k$-krat, $k = 0, 1, 2, \ldots, n$. \\

\noindent $A_n(k)$ je \textit{disjunktivna unija} $\mathlarger{\binom{n}{k}}$ dogodkov, da se $A$ zgodi na predpisanih $k$ mestih; na ostalih pa $A^C$. Verjetnost le teh dogodkov je $p^k \cdot q^{n-k}$. Zato velja \textit{Bernoullijeva formula}
$$\mathbb{P}_n(k) = \binom{n}{k} \cdot p^k \cdot q^{n-k}.$$

\end{definicija}
\vspace{0.5cm}

\begin{trditev}[Aproksimativni formuli za $\mathbb{P}_n(k)$]

~\\
\begin{enumerate}
	\item[a)] \textsc{Poissonova formula}: če je $p$ blizu $0$ in $n$ velik, potem je
	$$\mathbb{P}_n(k) \approx \frac{\lambda^k}{k!} \cdot e^{-\lambda}, ~\text{kjer je}~ \lambda = n \cdot p$$
	\item[b)] \textsc{Laplaceova formula}: za velike $n$ velja
	$$\mathbb{P}_n(k) \approx \frac{1}{\sqrt{2 \pi n p q}} \cdot e^{-\frac{(k-np)^2}{2npq}}$$
\end{enumerate}

\end{trditev}
\vspace{0.5cm}

% *************************************************************************************************

\pagebreak

% #################################################################################################

\section{SLUČAJNE SPREMENLJIVKE}
\vspace{0.5cm}

\begin{definicija}

\textit{Realna slučajna spremenljivka} na verjetnostnem prostoru $(\Omega, \mathcal{F}, p)$ je funkcija $X: \Omega \rightarrow \mathbb{R}$ z lastnostjo, da je za $\forall x \in \mathbb{R}$ množica $\{ \omega \in \Omega \mid X(\omega) \leq x \}$ v $\mathcal{F}$, se pravi je dogodek.\\

\noindent Oznaka:
$$\{ \omega \in \Omega \mid X(\omega) \leq x \} ~\equiv~ X^{-1}((-\infty, x]) ~\equiv~ (X \leq x)$$

\end{definicija}
\vspace{0.5cm}

\begin{definicija}[Porazdelitvena funkcija slučajne spremenljivke]

Funkcija $F_X: \mathbb{R} \rightarrow \mathbb{R}$, definirana s $F_X(x) = \mathbb{P}(X \leq x)$ se imenuje \textit{porazdelitvena funkcija slučajne spremenljivke $x$}.

\end{definicija}
\vspace{0.5cm}

\begin{trditev}[Lastnosti porazdelitvene funkcije $F = F_X$]
~\\
\begin{enumerate}
	\item $0 \leq F(x) \leq 1$ za $\forall x \in \mathbb{R}$
	\item  $F$ je naraščajoča funkcija:
	$$x_1 < x_2 \Rightarrow F(x_1) \leq F(x_2)$$
	\item $\mathlarger{\lim_{x \rightarrow \infty} F(x) = 1}$, $\mathlarger{\lim_{x \rightarrow -\infty} F(x) = 0}$
	\item $F$ je \textit{z desne zvezna}, torej $F(x+) = F(x)$ za $\forall x \in \mathbb{R}$, kjer je $\mathlarger{F(x+) = \lim_{h \searrow 0} F(x+h)}$ desna limita.
	\item $F(x-) = \mathbb{P}(X < x)$ 
	\begin{align*}
	\mathbb{P}(x_1 < X \leq x_2) &= F(x_2) - F(x_1) \\
	\mathbb{P}(x_1 < X < x_2) &= F(x_2-) - F(x_1) \\
	\mathbb{P}(x_1 \leq X \leq x_2) &= F(x_2) - F(x_1-) \\
	\mathbb{P}(x_1 \leq X < x_2) &= F(x_2-) - F(x_1-) 
	\end{align*}
\end{enumerate}
\end{trditev}
\vspace{0.5cm}

\begin{definicija}[Diskretne slučajne spremenljivke]

Slučajna spremenljivka je \textit{diskretno porazdeljena}, če je njena zaloga vrednosti \textit{končna} ali \textit{števna končna} množica. Naj bo $\{ x_1, x_2, \ldots \}$ zaloga vrednosti. Vpeljemo verjetnostno funkcijo $p_k = \mathbb{P}(X = x_k)$, $k = 1, 2, 3, \ldots.$ Ker so $\{ (X = x_k) \}_k$ \textit{poln} sistem dogodkov, je 
$$\sum_k p_k = 1.$$
$X$ lahko zapišemo s shemo 
$X: \begin{pmatrix}
	x_1 & x_2 & \cdots & x_n \\
	p_1 & p_2 & \cdots & p_n
\end{pmatrix}$. Porazdelitvena funkcija:
$$F(x) = \mathbb{P}(X \leq x) = \mathbb{P}\left( \bigcup_{k ~\mid~ x_k \leq x} (X = x_k) \right) = \sum_{k ~\mid~ x_k \leq x} p_k.$$ \\

\noindent Pogoste diskretne porazdelitve:
\begin{enumerate}
	\item \textsc{Enakomerna porazdelitev} na $n$ točkah $x_1, \ldots, x_n$:
	$$X: \begin{pmatrix}
		x_1 & x_2 & \cdots & x_n \\
		\frac{1}{n} & \frac{1}{n} & \cdots & \frac{1}{n}
	\end{pmatrix}$$
	\item \textsc{Bernoullijeva porazdelitev}, $Ber(p)$, $p \in (0, 1)$:
	$$X: \begin{pmatrix}
		0 & 1 \\
		(1-p) & p
	\end{pmatrix}$$
	Indikatorska funkcija:
	$$1_A(\omega) = \begin{cases}
	1 ; ~\omega \in A \\
	0 ; ~\omega \notin A
	\end{cases}$$
	\item \textsc{Binomska porazdelitev}, $Bin(n, p)$, $n \in \mathbb{R}$, $p \in (0, 1)$:
	$$X: \begin{pmatrix}
		0 & 1 & 2 & \cdots & n \\
		p_0 & p_1 & p_2 & \cdots & p_n
	\end{pmatrix},$$
	kjer za $k = 0, 1, \ldots, n$ velja
	$$p_k ~=~ \mathbb{P}(X = k) ~=~ \binom{n}{k} p^k (1 - p)^{n-k}.$$
	\item \textsc{Poisonnova porazdelitev}, $Poi(\lambda)$, $\lambda = n \cdot p > 0$:
	$$p_k ~=~ \mathbb{P}(X = k) ~=~ \mathbb{P}_n(k) ~=~ \frac{\lambda^k}{k!} \cdot e^{-\lambda}, ~k = 0, 1, \ldots$$
	Sledi:
	$$\sum_{k=0}^{\infty} ~=~ e^{-\lambda} \sum_{k=0}^{\infty} \frac{\lambda^k}{k!} ~=~ e^{-\lambda} \cdot e^{\lambda} ~=~ 1.$$
\end{enumerate}

\end{definicija}
\vspace{0.5cm}

\end{document}
